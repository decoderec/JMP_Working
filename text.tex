

\documentclass[9pt]{beamer}
\mode<presentation>
\linespread{1.5}

\usetheme{Madrid}
%\usetheme {CambridgeUS}%{Goettingen}%{Berkeley}%{Montpellier}%{Antibes}%{Dresden}%{Madrid}%{Dresden}%{Darmstadt}%{Warsaw}%{Pittsburgh}
\usefonttheme{{structuresmallcapsserif}}%{structureitalicserif}%{structurebold}%{structuresmallcapsserif}%{professionalfonts}
%\useoutertheme[subsection=false]{smoothbars}
\usepackage[english]{babel}
%\usepackage[latin1]{inputenc}
\usepackage{hyperref}
%\usecolortheme{beaver}
%\usecolortheme{crane}

%\setbeamercovered{transparent}
%\newcommand{\semitransp}[2][35]{\color{fg!#1}#2}

\usecolortheme[RGB={0,23,105}]{structure}

\usepackage{latexsym}
\usepackage{amsmath}
\usepackage{times}
%\usepackage{MinionPro}
\usepackage{hyperref}
\usepackage{tikz}
\usepackage{verbatim}
\usepackage{natbib}
\usepackage{color, colortbl}
\usepackage{appendix}
\usepackage{ulem}
\usepackage{amsmath,amsthm}
\usepackage{bbm}
\usetikzlibrary{arrows,shapes}


\newtheorem{proposition}{Proposition}[section]
%\newtheorem{definition}{Definition}[section]
\newtheorem{assumption}{Assumption}[section]
\newtheorem{conjecture}{Conjecture}[section]
\newtheorem*{observation}{Observation}


\definecolor{Gray}{gray}{0.9}


\pgfdeclarelayer{background}
\pgfsetlayers{background,main}

\tikzstyle{vertex}=[circle, draw
%,fill=black!25
,minimum size=12pt
,inner sep=0pt
]
\tikzstyle{selected vertex} = [vertex, fill=red!24]
\tikzstyle{unknown vertex} = [vertex, fill=black!25]
\tikzstyle{edge} = [draw,thick,-]
\tikzstyle{weight} = [font=\small]
\tikzstyle{selected edge} = [draw,line width=5pt,-,red!50]
\tikzstyle{ignored edge} = [draw,line width=5pt,-,black!20]







\title{Coordination in Social Networks}
\subtitle{Communication by Actions}
\author{Chun-Ting Chen}


\begin{document}

%\maketitle


\section{Introduction}
\subsection{Motivation}






\section{Model}
\subsection{Model}

\frame{
  \frametitle{Model}
Time line

\begin{enumerate}


\item There is a \alert{fixed}, \alert{finite}, \alert{connected}, \alert{undirected}, and \alert{commonly known} network.
\item Players of two types--- \textit{R} or \textit{I} ---chosen by nature according to a probability distribution.
\begin{itemize}
\item \textit{R}: Rebel; \textit{I}: Inert
\end{itemize}
\item Types are then fixed over time.
\item Players play a stage game--- a collective action ---infinitely repeatedly with common discount factor.
\end{enumerate}



}


\begin{frame}
  \frametitle{Model}

What player can/cannot observe

\begin{itemize}
\item Players can observe own/neighbors' \textcolor{blue}{types} and \textcolor{blue}{actions}, but not others'.
\item Pay-off is hidden.

\end{itemize}




\end{frame}


\begin{frame}
  \frametitle{Model}

  \begin{itemize}

  \item Stage game---\alert{$k$}-threshold game: a \textcolor{blue}{protest} (~[Chwe 2000])




\begin{itemize}
\item R-type's action set$=\{\textbf{1},\textbf{0}\}$
\item I-type's action set$=\{\textbf{0}\}$
\item Pay-offs for R-type:
\begin{table}[h]
\begin{tabular}{llll}
$u_{R}(a_{i},a_{-_i})$ & $=$ & 1 & if $a_{i}=\textbf{1}$ and $\#\{j:a_{j}=\textbf{1}\}\geq {k}$ \\
$u_{R}(a_{i},a_{-i})$ & $=$ & -1 & if $a_{i}=\textbf{1}$ and $\#\{j:a_{j}=\textbf{1}\}< {k}$ \\
\\
$u_{R}(a_{i},a_{-i})$ & $=$ & 0 & if $a_{i}=\textbf{0}$ 
\end{tabular}

\end{table}
\end{itemize}
  

 \end{itemize}

\end{frame}



\begin{frame}
  \frametitle{Static ex-post Pareto efficient outcome}



\begin{table}[h]
\begin{tabular}{ll}
Type profile & Static ex-post efficient outcome \\
\hline
At least $k$ R-types exist & All R-types play \textbf{1}  \\
Otherwise &  All R-types play \textbf{0} 
\end{tabular}
\end{table}
\end{frame}







\begin{frame}
  \frametitle{APEX equilibrium}

\textcolor{blue}{APEX} (\textit{approaching ex-post efficient}) equilibrium

\begin{definition}[APEX strategy]
An equilibrium is APEX $\Leftrightarrow$ 
\[\text{\alert{$\forall\theta$},  there is a finite time $T^{\theta}$}\] 
such that the actions in the equilibrium path repeats the static ex-post efficient outcome after $T^{\theta}$. 
\end{definition}

\end{frame}










%\begin{frame}
%  \frametitle{Notations}
%
%Notations:
%\begin{itemize}
%
%\item $\alert{\theta_{G_i}}\in \Theta_{G_i}$: $i$'s private information about the state. 
%\item $\alert{h^{m}_{G_i}}\in H^m_{G_i}$: the history of actions observed by $i$ up to period $m$.
%\item \alert{$\Theta_{G_i}\times H^m_{G_i}$}: $i$'s observation up to time $m$.
%\pause
%\item $h^m$: a sequence of players' actions up to period $m$.
%\item $h$: an infinite sequence of players' actions. 
%
%
%
%
% 
%\end{itemize}
%
%\end{frame}
%
%
%
%
%\begin{frame}
%  \frametitle{APEX strategy path}
%Notations:
%\begin{itemize}
%\item $\tau_i:\Theta_{G_i}\times \bigcup^{\infty}_{m=0} H^{m}_{G_i} \rightarrow A_{\theta_i}$, $i$'s strategy.
%\item $\tau=(\tau_1,...,\tau_i,...,\tau_n)$: a strategy profile. 
%\item $h^{\tau}_{\theta}$ : a history generated by $\tau$ given $\theta$.
%\item $\tau$-path: $\{h^{\tau}_{\theta}\}_{\theta\in \Theta}$
%\end{itemize}
%
%\begin{definition}
%The $\tau$-path is \textbf{approaching ex-post efficient} (\textcolor{blue}{\textit{APEX}}) $\Leftrightarrow$ 
%\[\text{$\forall\theta$,  there is a finite time $T^{\theta}$}\] 
%such that the actions after $T^{\theta}$ in $h^{\tau}_{\theta}$ repeats the static ex-post efficient outcome.
%\end{definition}
%
%
%\end{frame}
%
%
%
%
%\begin{frame}
%  \frametitle{Equilibrium Concept}
%
%Notations:
%\begin{itemize}
%
%\item $\beta^{\pi,\tau}_i(\theta|h^{m}_{G_i})$: $i$'s belief for a $\theta$ at period $m$ given $\pi,\tau$.
%\item $\phi_{G_i}: H^m\rightarrow H^m_{G_i}$: the projection mapping a $h^m$ to $h^m_{G_i}$.
%\end{itemize}
%
%\begin{definition}
%$h^m_{G_i}$ is \textbf{reached} by $\tau$ iff there is a pair $(\theta,h^m)$ such that $h^m$ is on the $\tau$-path, and $h^m_{G_i}=\phi_{G_i}(h^m)$. 
%\end{definition}
%
%
%\end{frame}
%
%
%\begin{frame}
%  \frametitle{Equilibrium Concept}
%
%
%\begin{definition}[weak sequential equilibrium]
%The pair $(\tau^{*},\beta^{*})$, 
%\begin{itemize}
%\item $\tau^{*}$: a strategy
%\item $\beta^{*}=\{\beta^{*,m}\}_m$: the belief system 
%\begin{itemize}
%\item $\beta^{*,m}_i$: $\Theta_{G_i}\times H^{m}_{G_i} \rightarrow \Delta (\Theta\times H^m)$
%\end{itemize}
%\end{itemize}
%
%, is a weak sequential equilibrium iff
%\begin{itemize}
%\item $\beta^{*,m}_i(\theta|h^{m}_{G_i})=\beta^{\pi,\tau^{*}}_i(\theta|h^{m}_{G_i})$ whenever $h^{m}_{G_i}$ is reached by $\tau^{*}$ for all $i$.
%\item $\tau^{*}$ is sequential rational given $\beta^{*}$.
%\end{itemize}
%
%
%
%
%
%\end{definition}
%
%
%
%
%\end{frame}
%
%
%\begin{frame}
%  \frametitle{Equilibrium Concept}
%
%
%\begin{definition}[Sequential equilibrium]
%A sequential equilibrium $(\tau^{*},\beta^{*})$  is a weak  sequential equilibrium and $\beta^{*}$ is fully consistent with $\tau^{*}$[Krep and Wilson].
%\end{definition}
%\begin{itemize}
%\item Fully consistent: 
%the $\beta^{*}$ is \textbf{``very very similar with''} that belief system induced by a \textbf{``very very little perturbed''} strategies around $\tau^{*}$.
%\end{itemize}
%\end{frame}
%
%\begin{frame}
%  \frametitle{APEX Equilibrium}
%
%\begin{itemize}
%\item Finally, let the ``(weak) APEX equilibrium'' be \textcolor{blue}{the (weak) sequential equilibrium in which the equilibrium path is APEX}.
%\item Does an APEX equilibrium exist?
%\end{itemize}
%
%\end{frame}
%


\frame{
  \frametitle{Result 1: APEX for $k=n$}

\begin{theorem}[\alert{$k=n$}]

If $k=n$, then an APEX sequential equilibrium exists whenever discount factor is sufficiently high.
\end{theorem}


}


\begin{frame}
  \frametitle{Definition for APEX for $k<n$}


\begin{definition}
$\theta$ has \textbf{strong connectedness}$\Leftrightarrow$ for every pair of R-types, there is a path consisting of R-types to connect them.
\end{definition}  

\begin{definition}
$\pi$ has \textbf{full support on strong connectedness}$\Leftrightarrow$ 
\[\text{$\pi(\theta)>0$ \textbf{if and only if} $\theta$ has strong connectedness.}\]
\end{definition}  

%Ex.:
%\begin{itemize}
%\item $\pi(\theta)>0$ if $\theta=$
%\begin{center}
%\begin{tikzpicture}[scale=1]
%    % Draw a 7,11 network
%    % First we draw the vertices
%    \foreach \pos/\name in {{(1,0)/S_1}, {(2,0)/S_2}, {(3,0)/S_3}}
%        \node[vertex] (\name) at \pos {$\name$};
%    % Connect vertices with edges 
%    \foreach \source/ \dest in {S_1/S_2, S_2/S_3}
%        \path[edge] (\source) -- (\dest) ;
%        
%\end{tikzpicture}
%%\text{ or }
%%\begin{tikzpicture}[scale=1]
%%    % Draw a 7,11 network
%%    % First we draw the vertices
%%    \foreach \pos/\name in {{(1,0)/S_1}, {(2,0)/S_2}, {(3,0)/S_3}}
%%        \node[vertex] (\name) at \pos {$\name$};
%%    % Connect vertices with edges 
%%    \foreach \source/ \dest in {S_1/S_2, S_2/S_3}
%%        \path[edge] (\source) -- (\dest) ;
%%        
%%\end{tikzpicture}
%\end{center}
%\item $\pi(\theta)=0$ if $\theta=$
%
%\begin{center}
%\begin{tikzpicture}[scale=1]
%    % Draw a 7,11 network
%    % First we draw the vertices
%    \foreach \pos/\name in {{(1,0)/S_1}, {(3,0)/S_3}}
%        \node[vertex] (\name) at \pos {$\name$};
%    % Connect vertices with edges 
%    \foreach \pos/\name in {{(2,0)/B_2}}
%    \node[unknown vertex] (\name) at \pos {$\name$};
%    
%    \foreach \source/ \dest in {S_1/B_2, B_2/S_3}
%        \path[edge] (\source) -- (\dest) ;
%        
%\end{tikzpicture}
%\end{center}
%
%\end{itemize}

\end{frame}





\frame{
  \frametitle{Result 2: APEX for $k<n$}

\begin{theorem}[\alert{$k< n$}]
If $k<n$, then if network is a \alert{tree}, if prior $\pi$ has \alert{full support on strong connectedness}, then an APEX WPBE {exists} whenever discount factor is sufficiently high.
\end{theorem}

}





\end{document}
