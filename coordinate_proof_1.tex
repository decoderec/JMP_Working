\documentclass[12pt,letter]{article}
\usepackage[DIV=14,BCOR=2mm,headinclude=true,footinclude=false]{typearea}
\renewcommand{\baselinestretch}{1.5} 
\usepackage{latexsym}
\usepackage{amsmath}

%\usepackage{MinionPro}
\usepackage{amssymb}

\usepackage{hyperref}
\usepackage{tikz}
\usepackage{verbatim}
\usepackage{natbib}
\usepackage{color, colortbl}
\usepackage{appendix}
\usepackage{amsmath,amsthm}

%\usepackage[]{figcaps}

\usepackage{wasysym}


\usetikzlibrary{arrows,shapes}

\definecolor{Gray}{gray}{0.9}

\newtheorem{result}{Result}
\newtheorem{theorem}{Theorem}
\newtheorem*{theorem*}{Theorem}
\newtheorem{conjecture}{Conjecture}[section]
\newtheorem{corollary}{Corollary}[section]
\newtheorem{lemma}{Lemma}[section]
\newtheorem*{lemma*}{Lemma}
\newtheorem{proposition}{Proposition}[section]
\newtheorem{definition}{Definition}[section]
\newtheorem{assumption}{Assumption}[section]


\theoremstyle{definition}
\newtheorem{example}{Example}

\theoremstyle{remark}
\newtheorem*{remark}{Remark}

\theoremstyle{claim}
\newtheorem{claim}{Claim}


\pgfdeclarelayer{background}
\pgfsetlayers{background,main}

\tikzstyle{vertex}=[circle, draw
%,fill=black!25
,minimum size=12pt
,inner sep=0pt
]
\tikzstyle{selected vertex} = [vertex, fill=red!24]
%\tikzstyle{unknown vertex} = [vertex, fill=black!25]
\tikzstyle{unknown vertex} = [vertex, fill=white]
\tikzstyle{edge} = [draw,thick,-]
\tikzstyle{weight} = [font=\small]
\tikzstyle{selected edge} = [draw,line width=5pt,-,red!50]
\tikzstyle{ignored edge} = [draw,line width=5pt,-,black!20]



%\linespread{1.5}

\begin{document}




\section{Model}
\label{sec:model}
%\subsection{Notations}
Given a finite set $X$, denote $\#X$ as its cardinality . 
%The square bracket $[]$ that follows a quantifier $\exists, \forall$ will be read as ``\textit{such that}''. For instance, $\exists a \in A [c\in A, c=a]$ will be read as ``exists $a$ in $A$ such that $c$ in A and $c$ is equal to $a$.'' .
%\subsection{Model}


There is a set of players $N=\{1,2,...,n\}$. They constitute a network $G=(V,E)$ so that the vertices are players ($V=N$) and an edge is a pair of them ($E$ is a subset of the set containing all two-element subsets of $N$). Throughout this paper, $G$ is assumed to be finite, commonly known, fixed, undirected, and connected.\footnote{A path in $G$ from $i$ to $j$ is a finite sequence $(l_1,l_2,...,l_L)$ without repetition so that $l_1=i$, $l_L=j$, and $\{l_q,l_{q+1}\}\in E$ for all $1\leq q<L$. $G$ is fixed if $G$ is not random, and $G$ is undirected if there is no order relation over each edge. $G$ is connected if, for all $i,j\in N$, $i\neq j$, there is a path from $i$ to $j$.}
For the sake of convenience, $G_i\equiv \{j:\{i,j\}\in E\}$.

Time is discrete with index $t\in\{0,1,...\}$. At $t=0$, the nature chooses a state $\theta\in \Theta=\{R,I\}^n$ once and for all according to a common prior $\pi$. $R$ and $I$ represent as Rebel and Inert respectively. For the sake of convenience, let $[R](\theta)$ be the set of Rebels given $\theta$ and the notion \textit{relevant information} indicate to the information about whether or not $\#[R](\theta)\geq k$.



\begin{definition}[Strong connectedness]
Given $G$, a state $\theta$ has strong connectedness if, for every pair of Rebels, there is a path consisting of Rebels to connect them.

\end{definition}  

In the language of graph theory, this definition is equivalent to: ``Given $G$, $\theta$ has strong connectedness if the induced graph by $[R](\theta)$ is connected.''

\begin{definition}[Full support on strong connectedness]
Given $G$, $\pi$ has full support on strong connectedness if 
\[\pi(\theta)>0\Leftrightarrow \text{ $\theta$ has strong connectedness }\] 
\end{definition}  


\subsubsection{Information hierarchy}
\label{sec:info}
The information hierarchy across Rebels at $t$ in $G$ is a tuple \[(\{G^{t}_i\}_{i\in N}, \{I^{t}_i\}_{i\in N}, R^t,\theta).\]
$G^{t}_i$ is meant to represent \textit{the extended neighbors} to represent the following. $j\in G^{t}_i$ if $j$ can be reached by $t$ consecutive edges from $i$ such that the endpoints of $t-1$ edges are both Rebels but the endpoints of the remaining one are $j$ and a Rebel; $I^{t}_i$ is interpreted as \textit{the extended Rebel neighbors}---the set of Rebels in $G^t_i$; $R^t$ is interpreted as \textit{the active Rebels}---those Rebels who are \textit{active} in the sense that their extended Rebel neighbors are not a subset their direct neighbors' extended Rebel neighbors. They are defined recursively: 

At $t=0$,
\[\text{if $\theta_i=I$, $G^{0}_i\equiv \emptyset$, $I^{0}_i\equiv \emptyset$.}\] 
\[\text{if $\theta_i=R$, $G^{0}_i\equiv \{i\}$, $I^{0}_i\equiv \{i\}$.}\] 
\[\text{$R^0\equiv [R](\theta)$.}\] 

At $t=1$,
\[\text{if $\theta_i=I$, $G^{1}_i\equiv \emptyset$, $I^{1}_i\equiv \emptyset$.}\] 
\[\text{if $\theta_i=R$, $G^{1}_i\equiv G_i$, $I^{0}_i\equiv G_i\cap R^0$.}\] 
\[\text{$R^1\equiv \{i\in R^0: \nexists j\in G_i \text{ such that }I^1_i\subseteq G^1_j\}$.}\] 

At $t>1$, 
\[\text{if $\theta_i=I$, $G^{t}_i\equiv \emptyset$, $I^{t}_i\equiv \emptyset$.}\] 
\[\text{if $\theta_i=R$, $G^{t}_i\equiv \bigcup_{j\in G_i} G^{t-1}_j$, $I^{t}_i\equiv \bigcup_{j\in G_i} I^{t-1}_j$.}\] 
\[\text{$R^t\equiv \{i\in R^{t-1}: \nexists j\in G_i \text{ such that }I^t_i\subseteq G^t_j\}$.}\]



For convince, also denote $I^{t+1}_{ij}=I^t_i\cap I^t_j$ if $j\in G_i\cap R^{t}$. It can be shown that $R^t\subset R^{t-1}$ by the following lemma. 
\begin{lemma}
\label{lemma_inclusion}
If the network is acyclic and if the $\theta$ has strong connectedness, then 
\[R^t\subset R^{t-1}\] for all $t\geq 1$.
\end{lemma}





\end{document}
