\documentclass[12pt,letter]{article}
\usepackage[DIV=14,BCOR=2mm,headinclude=true,footinclude=false]{typearea}
\renewcommand{\baselinestretch}{1.15} 
\usepackage{latexsym}
\usepackage{amsmath}
%\usepackage{MinionPro}
\usepackage{hyperref}
\usepackage{tikz}
\usepackage{verbatim}
\usepackage{natbib}
\usepackage{color, colortbl}
\usepackage{appendix}
\usepackage{amsmath,amsthm}


\usepackage{wasysym}
\usepackage{amssymb}

\usetikzlibrary{arrows,shapes}

\definecolor{Gray}{gray}{0.9}

\newtheorem{result}{Result}
\newtheorem{theorem}{Theorem}
\newtheorem{conjecture}{Conjecture}[section]
\newtheorem{corollary}{Corollary}[section]
\newtheorem{lemma}{Lemma}[section]
\newtheorem{proposition}{Proposition}[section]
\newtheorem{definition}{Definition}[section]
\newtheorem{assumption}{Assumption}[section]


\theoremstyle{definition}
\newtheorem{example}{Example}[section]

\theoremstyle{remark}
\newtheorem*{remark}{Remark}

\theoremstyle{claim}
\newtheorem{claim}{Claim}


\pgfdeclarelayer{background}
\pgfsetlayers{background,main}

\tikzstyle{vertex}=[circle,fill=black!25,minimum size=12pt,inner sep=0pt]
\tikzstyle{selected vertex} = [vertex, fill=red!24]
\tikzstyle{edge} = [draw,thick,-]
\tikzstyle{weight} = [font=\small]
\tikzstyle{selected edge} = [draw,line width=5pt,-,red!50]
\tikzstyle{ignored edge} = [draw,line width=5pt,-,black!20]


%\linespread{1.5}

\begin{document}
%\fontsize{12}{20pt}\selectfont

\title {Coordination in Social Networks}
\author {by Chun-Ting Chen}

\maketitle

\begin{abstract}

This paper studies a collective action problem in a setting of discounted repeated coordination games in which players know their neighbors'  inclination to participate as well as monitor their neighbors' past actions. I define \textit{strong connectedness} to characterize those states in which, for every two players who incline to participate, there is a path consisting of players with the same inclination to connect them.  Given that the networks are fixed, finite, connected, commonly known, undirected and without cycles, I show that if the priors have full support on the strong connectedness states, there is a (weak) sequential equilibrium in which the ex-post efficient outcome repeats after a finite time $T$ in the path when discount factor is sufficiently high. This equilibrium is constructive and does not depend on public or private signals other than players' actions.




\end{abstract}





\end{document}
