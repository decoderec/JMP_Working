\documentclass[12pt,letter]{article}
\usepackage[DIV=14,BCOR=2mm,headinclude=true,footinclude=false]{typearea}
\renewcommand{\baselinestretch}{1.15} 
\usepackage[affil-it]{authblk}
\usepackage{latexsym}
\usepackage{amsmath}
%\usepackage{MinionPro}
\usepackage{hyperref}
\usepackage{tikz}
\usepackage{verbatim}
\usepackage{natbib}
\usepackage{color, colortbl}
\usepackage{appendix}
\usepackage{amsmath,amsthm}


%\usepackage{wasysym}
%\usepackage{amssymb}

\usetikzlibrary{arrows,shapes}

\definecolor{Gray}{gray}{0.9}

\newtheorem{result}{Result}
\newtheorem{theorem}{Theorem}
\newtheorem{conjecture}{Conjecture}[section]
\newtheorem{corollary}{Corollary}[section]
\newtheorem{lemma}{Lemma}[section]
\newtheorem{proposition}{Proposition}[section]
\newtheorem{definition}{Definition}[section]
\newtheorem{assumption}{Assumption}[section]


\theoremstyle{definition}
\newtheorem{example}{Example}[section]

\theoremstyle{remark}
\newtheorem*{remark}{Remark}

\theoremstyle{claim}
\newtheorem{claim}{Claim}


\pgfdeclarelayer{background}
\pgfsetlayers{background,main}

\tikzstyle{vertex}=[circle,fill=black!25,minimum size=12pt,inner sep=0pt]
\tikzstyle{selected vertex} = [vertex, fill=red!24]
\tikzstyle{edge} = [draw,thick,-]
\tikzstyle{weight} = [font=\small]
\tikzstyle{selected edge} = [draw,line width=5pt,-,red!50]
\tikzstyle{ignored edge} = [draw,line width=5pt,-,black!20]


\linespread{1.2}

\begin{document}
%\fontsize{12}{20pt}\selectfont

\title {Teaching Statement}
\author{Chun-Ting Chen%
  \thanks{\texttt{cuc230@psu.edu}}}
%\affil{The Pennsylvania State University}
\date{}

%\maketitle

\section*{Teaching Statement}

Historical examples and empirical findings are critical components of my Economics teaching. The vivid examples in \textit{The Worldly Philosophers}~ by Robert L. Heilbroner introduced me to how economists think about production and consumption and was a major reason for my pursuing a career in Economics.  I believe that student interest in economic theory is stimulated by empirical applications and examples. How excited I was when I discovered a significant coefficient in my first empirical methods course. Moreover, students' questions and feedback can improve their teacher's understanding of Economics. For instance, an classroom experiment may lead to a new direction for research.

My research informs my teaching. For instance, suppose I were teaching decision theory, say the expected utility theorem. I might conduct some variant of the Allais Paradox experiment before getting into the details. Only after students are made aware of the strong assumptions behind the theory would I introduce the main results. Suppose I were teaching industrial organization to undergraduate students and suppose the day's topic were price discrimination. I might discuss something from the business news to make students understand what the theory is meant to explain. Just as a thorough understanding of the literature is indispensable when conducting research, I believe students better understand what economic theories are used for when given the relevant context.

As a faculty member I will practice teaching techniques which I have found particularly useful as a student over the years. One such technique is drawing graphs to derive theoretical intuition. When I was an undergraduate student in Civil Engineering, Mohr's Circle left me a with a deep impression of how useful a clever chart can be in the understanding difficult concepts. Many Economic theorems involve a continuum. Using a graph to represent mathematical objects typically gives me strong intuition about the underlying maximization problems. In the classroom, I have found that a short motivating theoretical result accompanied by lecture notes and supplemental material is the best way to hold my attention and my interaction. Keeping that in mind, I am committed to help students to actively participate in the classroom.


\end{document}
